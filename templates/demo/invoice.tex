<?lsmb#   This is a comment block; it's ignored by the template engine.

   Version:  1.3
   Date:     2024-08-14
   File:     invoice.tex
   Set:      demo

Template version numbers are explicitly not aligned across templates or
releases. No explicit versioning was applied before 2021-01-04.

Version   Changes
1.1       Fix spelling mistake (acrue -> accrue)
1.2       Merged xelatex targetting templates with those targetting pdflatex
1.3       Extended with 'shiptoattn' (at the attention of)


The "$FORMAT($PROCESSOR)" below evaluates to e.g. "pdf(pdflatex)".

-?>
<?lsmb FILTER latex { format="$FORMAT($PROCESSOR)" };
       # Uncomment the next line to overrule the value from System > Defaults;
       # The value should be without the 'paper' suffix. E.g. "a4" for "a4paper"
       # SETTINGS.papersize = "<your-format>";
       INCLUDE "preamble.tex"; -?>

\setlength\LTleft{0pt}
\setlength\LTright{0pt}

\begin{document}

\pagestyle{myheadings}
\thispagestyle{empty}

\ifpdftex
  %% (pdf)latex
  \fontfamily{cmss}\fontsize{10pt}{12pt}\selectfont
\fi

<?lsmb BLOCK multiline -?>
\begin{minipage}{2in}
\medskip
\raggedright
<?lsmb string ?>
\end{minipage}
<?lsmb- END -?>

\newsavebox{\ftr}
\sbox{\ftr}{
  \parbox{\textwidth}{
  \tiny
  \rule[1.5em]{\textwidth}{0.5pt}
<?lsmb text('Payment due NET [_1] Days from date of Invoice.', terms) ?>
<?lsmb text('Interest on overdue amounts will accrue at the rate of 12% per annum starting from [_1] until paid in full. Items returned are subject to a 10% restocking charge.', duedate) ?>
<?lsmb text('A return authorization must be obtained from [_1] before goods are returned. Returns must be shipped prepaid and properly insured. [_1] will not be responsible for damages during transit.', company) ?>
  }
}

<?lsmb INCLUDE letterhead ?>

\markboth{<?lsmb company ?>\hfill <?lsmb invnumber ?>}{<?lsmb company ?>\hfill <?lsmb invnumber ?>}

\vspace*{0.5cm}

\parbox[t]{.5\textwidth}{
\textbf{To}
\vspace{0.3cm}

<?lsmb name ?>

<?lsmb address1 ?>

<?lsmb address2 ?>

<?lsmb city ?>
<?lsmb IF state -?>
\hspace{-0.1cm}, <?lsmb state ?> <?lsmb END ?> <?lsmb zipcode ?>

<?lsmb country ?>

\vspace{0.3cm}

<?lsmb IF contact ?>
<?lsmb contact ?>
\vspace{0.2cm}
<?lsmb END ?>

<?lsmb IF customerphone ?>
Tel: <?lsmb customerphone ?>
<?lsmb END ?>

<?lsmb IF customerfax ?>
Fax: <?lsmb customerfax ?>
<?lsmb END ?>

<?lsmb email ?>
}
\parbox[t]{.5\textwidth}{
\textbf{<?lsmb text('Ship To') ?>}
\vspace{0.3cm}

<?lsmb shiptoname ?>

<?lsmb IF shiptoattn -?>
<?lsmb shiptoattn ?>

<?lsmb END -?>
<?lsmb shiptoaddress1 ?>

<?lsmb shiptoaddress2 ?>

<?lsmb shiptocity ?>
<?lsmb IF shiptostate -?>
\hspace{-0.1cm}, <?lsmb shiptostate ?><?lsmb END ?> <?lsmb shiptozipcode ?>

<?lsmb shiptocountry ?>

\vspace{0.3cm}

<?lsmb IF shiptocontact ?>
<?lsmb shiptocontact ?>
\vspace{0.2cm}
<?lsmb END ?>

<?lsmb IF shiptophone ?>
Tel: <?lsmb shiptophone ?>
<?lsmb END ?>

<?lsmb IF shiptofax ?>
Fax: <?lsmb shiptofax ?>
<?lsmb END ?>

<?lsmb shiptoemail ?>
}
\hfill

\vspace{1cm}

\textbf{\MakeUppercase{<?lsmb
    IF reverse;
       text('Credit Invoice');
    ELSE;
       text('Invoice');
    END; ?>}}
\hfill

\vspace{1cm}

\begin{tabularx}{\textwidth}{*{7}{|X}|} \hline
  \textbf{<?lsmb text('Invoice #') ?>} & \textbf{<?lsmb text('Date') ?>} 
      & \textbf{<?lsmb text('Due') ?>} & \textbf{<?lsmb text('Order #') ?>}
      & \textbf{<?lsmb text('Salesperson') ?>} 
      & \textbf{<?lsmb text('Shipping Point') ?>} 
      & \textbf{<?lsmb text('Ship via') ?>} \\ [0.5em]
  \hline
  <?lsmb invnumber ?> & <?lsmb invdate ?> & <?lsmb duedate ?> & <?lsmb ordnumber ?> & <?lsmb employee ?>
  & <?lsmb shippingpoint ?> & <?lsmb shipvia ?> \\
  \hline
\end{tabularx}

\vspace{1cm}

\begin{longtable}{@{\extracolsep{\fill}}r|llcrlrr|r}

  \textbf{<?lsmb text('Item') ?>} 
  & \textbf{<?lsmb text('Number') ?>}
  & \textbf{<?lsmb text('Description') ?>} 
  & \textbf{<?lsmb text('Delivery') ?>} 
  & \textbf{<?lsmb text('Qty') ?>} 
  & \textbf{<?lsmb text('Unit') ?>} 
  & \textbf{<?lsmb text('Price') ?>} 
  &  \textbf{<?lsmb text('Disc %') ?>} 
  & \textbf{<?lsmb text('Amount') ?>} \\
\hline
\endhead
<?lsmb FOREACH number ?>
<?lsmb lc = loop.index ?>
  <?lsmb runningnumber.${lc} ?> & 
  <?lsmb number.${lc} ?> & 
  <?lsmb INCLUDE multiline string = item_description.${lc} ?> & 
  <?lsmb deliverydate.${lc} ?> &
  <?lsmb qty.${lc} ?> & 
  <?lsmb unit.${lc} ?> &
  <?lsmb sellprice.${lc} ?> &
  <?lsmb discountrate.${lc} ?> &
  <?lsmb linetotal.${lc} ?> \\
<?lsmb END ?>
<?lsmb IF tax ?>
\hline \hline
\multicolumn{8}{r|}{<?lsmb text('Subtotal') ?>} & <?lsmb subtotal ?> \\*
<?lsmb FOREACH tax ?>
<?lsmb lc = loop.index ?>
\multicolumn{8}{r|}{<?lsmb taxdescription.${lc} 
                    ?>  on <?lsmb taxbase.${lc} ?> }
 & <?lsmb tax.${lc} ?> \\*
<?lsmb END ?>
<?lsmb END ?>
\hline \hline
\multicolumn{8}{r|}{<?lsmb text('Total') ?>} & <?lsmb invtotal ?> \\*

<?lsmb IF paymentdate.0 ?>
\multicolumn{8}{r|}{ <?lsmb text('Paid') ?> } & - <?lsmb paid ?> \\*
  \hline
  \hline
\multicolumn{8}{r|}{<?lsmb text('Balance') ?>} & <?lsmb total ?>\\
<?lsmb END ?>

\end{longtable}


\parbox{\textwidth}{

\vspace{0.2cm}

\hfill

\vspace{0.3cm}

<?lsmb text_amount ?> ***** <?lsmb decimal ?>/100
\hfill
<?lsmb text('All prices in [_1].', currency) ?>

\vspace{12pt}
<?lsmb FOREACH P IN notes.split('\n\n+') ?>
<?lsmb P ?>\medskip

<?lsmb END ?>
}

\vfill

<?lsmb IF paymentdate.0 ?>
\begin{tabularx}{10cm}{@{}lXlr@{}}
  \textbf{<?lsmb text('Payments') ?>} & & & \\
  \hline
  \textbf{<?lsmb text('Date') ?>} & & \textbf{<?lsmb text('Source') ?>} 
  & \textbf{<?lsmb text('Amount') ?>} \\
<?lsmb FOREACH payment ?>
<?lsmb lc = loop.index ?>
  <?lsmb paymentdate.${lc} ?> & <?lsmb paymentaccount.${lc} ?> & <?lsmb paymentsource.${lc} ?> & <?lsmb payment.${lc} ?> \\
<?lsmb END ?>
\end{tabularx}
<?lsmb END ?>

<?lsmb FOR FILE IN file_list ?>
\includegraphics[scale=0.3]{<?lsmb file_path _ '/' _ FILE.file_name ?>}
<?lsmb END ?>

\vspace{1cm}

\centerline{\textbf{<?lsmb text('Thank You for your valued business!') ?>}}

\rule{\textwidth}{0.5pt}

\usebox{\ftr}

\end{document}
<?lsmb END -?>
